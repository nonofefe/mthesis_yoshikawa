\chapter{序論}
グラフ構造を持つデータの分析は近年の重要な研究分野の一つとされている。グラフとは「ノード」と、二つのノード間を結ぶ「エッジ」から構成されるデータ構造である。従来は、グラフ構造を持つデータの解析は情報量の多さやデータの複雑性から多くは研究されていなかった。しかし近年TwitterやFacebookなどのSNSの普及によりネットワーク上でのコミュニケーションが可能になったこと\cite{FreemanSocialNetwork}や、化合物の物性推定の重要性が高まったこと\cite{duvenaud2015convolutional}などにより、グラフ構造を含むデータの分析が注目を集めている。

このタスクを解決するためにグラフベース正則化\cite{belkin2004regularization}というアプローチを用いたLabel Propagation\cite{zhu2003semi}やLabel Spreading\cite{zhou2004learning}などの手法が古くに提案された。この手法はグラフのエッジに注目し、隣接ノード同士は等しいラベルを持つ可能性が高いという経験に基づいた手法である。次にグラフエンべディング(graph embedding)というノードやエッジなどをグラフ構造を保持したまま低次元ベクトル空間に埋め込む手法が提案された。グラフエンべディングにはランダムウォークを用いるモデル\cite{Perozzi2014deep}や、エンべディングした特徴量からグラフを再構成するモデル\cite{tang2015line}などがある。

そして第3のアプローチとしてspectral convolution\cite{bruna2013spectral}というグラフ構造に対して畳み込みを演算を行う手法が提案された。この手法は音声のフィルタリングと同様に、グラフ信号に対してフーリエ変換を行うことでグラフ構造での畳み込み演算を行う手法である。この手法の畳み込み演算は数学的な理論に基づいて行うことができるが、計算量が大きくなってしまう問題がある。ニューラルネットワークをグラフに適用し学習するGraph Neural Network(GNN)はその計算量の問題を解決できる手法として提案された。GNNはノードの特徴量だけでなく、グラフ構造を学習することができる。その中でも、1層で距離1の隣接ノードの特徴量の畳み込みを行うGraph Convolutional Network(GCN)\cite{kipf2016GCN}が、半教師ありノード分類やリンク予測などの様々なタスクで高い精度を示している。

GCNは学習の際に、特徴量に欠損値を含まないことを前提とした手法である。しかし実世界において欠損値を含むデータは数多く存在する。例えば(1)人的なミス、(2)センサーによるエラー入力、(3)任意回答における未入力項目を含むアンケート、(4)ビッグデータの情報欠損などにより、欠損値を含むデータを得られることがある。

欠損値を含むグラフデータを用いて学習する方法は、(1)欠損値を除去して学習する方法、(2)代入法を用いて欠損値を補完し、得られた欠損値を含まない特徴量に既存の機械学習方法を用いて学習する方法、(3)欠損値を含む特徴量をそのまま用いて学習する手法の3種類が存在する。従来の代入法の多くは、グラフ構造を無視して機械学習手法により欠損値を穴埋めしているために予測精度が低くなる可能性があるという問題がある。

本研究ではこのような欠損値を含むグラフデータに対して、グラフ構造を用いて欠損値補完を行うことで、GCNを用いた予測精度を向上させる手法を提案する。ノードの欠損値を近隣ノードの特徴量を平均して補完する近隣平均補完を用いたGCN\_neighborモデルと、欠損値を再帰的に補完する近隣再帰補完を用いたGCN\_recursiveモデルを提案手法とした。提案手法はグラフ構造を用いて欠損値補完を用いることで、GCNを用いた予測に適した特徴量を扱える強みがある。
  
提案手法の有効性を示すために半教師ありノード分類とリンク予測の2つのタスクを実験として行った。実験ではノードの特徴量は欠損値の割合を10\%から90\%まで10\%刻みで変化させて予測精度を調べた。実験の結果、主に欠損率が高い場合に、提案手法は既存手法と比べて高い精度を得ることを確認した。また提案手法におけるハイパーパラメータである再帰回数が十分であることを実験を通して検討した。

本論文は全6章から構成される。第2章では本研究で利用する従来の手法や関連研究について述べる。第3章では本研究における提案手法について具体的に述べる。第4章では論文引用ネットワークや共購買ネットワークを用いた実験を行う。第5章では本論文の結論と今後の課題について述べる。
