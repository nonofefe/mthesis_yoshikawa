\chapter{概要}
近年TwitterやFacebookなどのSNSの普及によりネットワーク上でのコミュニケーションが可能になったことや、化合物の物性推定の重要性が高まったなどにより、グラフ構造を含むデータの分析が注目を集めている。グラフとは「ノード」と、二つのノード間を結ぶ「エッジ」から構成されるデータ構造である。従来の研究において、グラフ構造とそれぞれのノードの特徴量を用いて予測するGraph Neural Network(GNN)が優れた成果を残している。その中でも、1層で距離1の隣接ノードの特徴量の畳み込みを行うGraph Convolutional Network(GCN)が、半教師ありノード分類やリンク予測などの様々なタスクで高い精度を示している。

GCNは学習の際に、特徴量に欠損値を含まないことを前提とした手法である。しかし実世界において欠損値を含むデータは数多く存在する。例えば人的ミス、センサーのエラー、任意回答における未入力項目を含むアンケートなどにより欠損値を含むデータを得られることがある。従来の欠損値を含むグラフデータに欠損値補完をする手法の多くは、グラフ構造を無視した機械学習手法により欠損値を穴埋めし、得られたデータを特徴量としてGCNなどのモデルを学習させる手法が一般的である。この方法は代入法にグラフ構造を用いないため予測精度が低くなる可能性がある。
  
本研究ではこのような欠損値を含むグラフデータに対して、グラフ構造を用いて欠損値補完を行うことで、GCNを用いた予測精度を向上させる手法を提案する。提案手法はグラフの近接ノードの情報を再帰的に集約し更新する手法を用いて欠損値を補完し、GCNを用いて予測するモデルにより構成されている。提案手法はグラフ構造を用いて欠損値補完を用いることで、GCNを用いて予測するために適した特徴量を扱えるため、予測精度を向上させることができた。
  
提案手法の有効性を示すために半教師ありノード分類とリンク予測の2つのタスクを実験として行った。実験ではノードの特徴量は欠損値の割合を10\%から90\%まで10\%刻みで変化させて予測精度を調べた。実験の結果、主に欠損率が高い場合に提案手法は既存手法と比べて高い精度を得ることを確認した。
