\chapter{結論}
本研究では欠損値を含むグラフデータに対してGCN(グラフ畳み込みネットワーク)を適用するために、グラフ構造を用いて欠損値を代入し補完する手法を提案した。提案手法としてノードの特徴量を近隣ノードの欠損していない特徴量の平均により補完するGCN\_neighborモデルと、近隣ノードの特徴量を用いて再帰的に補完するGCN\_recursiveモデルを定義し実験を行った。実験において提案手法のGCNは2層GCNを用い勾配降下法で学習するモデルを用いた。実験ではノード分類とリンク予測のタスクを欠損率を10\%から90\%まで変化させたときの精度を調べた。実験の結果、多くのデータセットや欠損率でGCN\_recursiveモデルが比較手法よりも高い精度を示すことが確認できた。また欠損率が高い場合でも提案手法は比較手法と比べて高い精度を維持するモデルであることが確認できた。

本研究の今後の課題として以下の3つが挙げられる。

1つ目の課題は欠損値補完に全ての特徴量の情報を用いていないことである。提案手法での欠損値補完のアルゴリズムは、特徴量の種類ごとにグラフ構造を用いて補完している。そのため、あるノードの欠損値補完はそのノードの別の特徴量に依存するものではない。この問題はそれぞれのノードごとの特徴量を共起しつつ欠損値補完をするモデルを作成することで解決できると考えられる。

2つ目の課題は欠損値補完の異なる方法の模索である。本研究での2つの提案手法はいずれも近隣ノードの特徴量の平均を用いて欠損値補完したものである。しかし近隣ノードの特徴量に偏りをかけて学習する方法やAttention機構を用いた方法、ノードエンべディングを用いてベクトル空間を用いて欠損値補完をする方法など様々なアプローチが考えられる。また異なるアプローチを組み合わせることにより精度のさらなる向上が見込めると思われる。

3つ目の課題は異なる欠損パターンにおける実験である。本研究のUniform randomly missing, Biased randomly missing, Structurally missingの3つの欠損パターンは全て人工的に作成させられたものでありMCARに分類される。そのためMNARやMARといった偶発的に生じた欠損パターンを考慮したモデルを考えることも今後の課題の1つである。
